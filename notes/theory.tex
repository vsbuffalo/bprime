\documentclass[11pt]{article}
\RequirePackage{fullpage}
\RequirePackage[font=small,labelfont=bf]{caption}
\RequirePackage{amsmath,amssymb,amsthm}
\RequirePackage{graphicx}
\RequirePackage[hidelinks]{hyperref}
\RequirePackage{subcaption}
\RequirePackage{wasysym}
\RequirePackage{authblk}
\RequirePackage{bm}
\RequirePackage{bbm}
% \RequirePackage[osf]{mathpazo}
\RequirePackage[bibstyle=authoryear,citestyle=authoryear-comp,maxbibnames=9,maxcitenames=1,backend=biber,natbib=true,uniquelist=false,hyperref=true]{biblatex}
\usepackage{color}
\usepackage{nicefrac}

\renewcommand{\P}{\mathbb{P}}
\renewcommand{\vec}[1]{\mathbf{#1}}
\newcommand{\E}{\mathbb{E}}
\newcommand{\V}{\text{V}}
\DeclareMathOperator{\var}{Var}
\DeclareMathOperator{\cov}{cov}

\addbibresource{biblio.bib}

\title{Statistical Methods}

\begin{document}
\maketitle


\section*{Likelihood}

The likelihood in Eyalshiv et al. (2016) has the form,

\begin{align}
  \log\mathcal{L}(\theta) = \sum_{v \in \mathcal{V}} \sum_{i \ne j \in \mathcal{S}} \log(P(O_{i,j}(v) | \theta))
\end{align}

where $\mathcal{V}$ is the set of putatively neutral sites, $\mathcal{S}$ is
the set of samples, and $\theta$ are the BGS parameters. The indicator variable
$O_{i,j}(v)$ is 1 if samples $i$ and $j$ are different at site $v$, and zero
otherwise. Thus as they specify in the paper, 

\begin{align}
  P(O_{i,j}(v) | \theta) = 
    \begin{cases}
      \pi(v | \theta), & O_{i,j}(v) = 1 \\
      1-\pi(v | \theta), & O_{i,j}(v) = 0 \\
    \end{cases}.
\end{align}

The total size of the set of samples $\mathcal{S}$ is $n_\mathcal{S} =
|\mathcal{S}|$. Assuming all sites are biallelic, we can simplify the inner
summation by counting the number of possible same and different pairwise
combinations. If a site $v$'s vector of allele counts is $[c_1, c_2]$, the
total number of pairwise combinations with the same alleles is

\begin{align}
  n_s(v) = {c_1 \choose 2} + {c_2 \choose 2}
\end{align}

and the number of different pairwise combinations is 

\begin{align}
  n_d(v) = n_T - n_s(v)
\end{align}

where $n_T = n_\mathcal{S} (n_\mathcal{S} - 1) / 2$  is the total number of
pairwise combinations across the sample set $\mathcal{S}$. Note that these
site-specific counts allow us to use allelic counts directly, and can vary
across sites. Our log-likelihood is then,

\begin{align}
  \ell(\theta) &= \sum_{v \in \mathcal{V}} \left[\log(\pi(v | \theta)) n_\text{D}(v) + \log(1-\pi(v | \theta)) n_\text{S}(v)\right].
\end{align}

In practice, we calculate these values across bins. For each bin, we treat
$\pi(v | \theta)$ as fixed, assuming that at this scale, the variation in
expected diversity across sites is minimal. For a particular chromosome, we
have two classes of sites: those included in the diversity calculation and
those ignored. The former sites are all putatively neutral and have reliably
called genotypes, and the other sites are possibly non-neutral or do have
reliably called genotypes. The total log-likelihood is the sum of bin likelihoods,
$\ell(b)$

\begin{align}
   \ell(\theta) =  \sum_b \ell(b | \theta)
\end{align}

The likelihood within a bin is then,

\begin{align}
  \ell(b | \theta)  &= \log(\bar{\pi}(b | \theta)) \sum_{v \in \mathcal{V}_b} n_D(v) + \log(1-\bar{\pi}(b | \theta)) \sum_{v \in \mathcal{V}_b} n_S(v)  \\
                               &= \log(\bar{\pi}(b | \theta)) Y_D(b) + \log(1-\bar{\pi}(b | \theta)) Y_S(b)
\end{align}
%
where the two sum terms as $Y_D(b)$ and $Y_S(b)$ are data reductions at the bin
level.

If the data are such that only polymorphic sites are considered, we can adapt
this by partitioning the set $\mathcal{V}$ of neutral sites into polymorphic
($\mathcal{P}$) and fixed sites ($\mathcal{F}$), i.e. $\mathcal{V} =
\mathcal{P} \cup \mathcal{F}$ and $\mathcal{P} \cap \mathcal{F} = \varnothing$.
For all $v \in \mathcal{F}$, $n_d(v) = 0$ and $n_s(v) = n_T$.

Then, 

\begin{align}
  \ell(\theta) &= \sum_{v \in \mathcal{V}} \left[\log(\pi(v | \theta)) n_\text{D}(v) + \log(1-\pi(v | \theta)) n_\text{S}(v)\right] \\
                  &= \sum_{v \in \mathcal{P}} \left[\log(\pi(v | \theta)) n_\text{D}(v) + \log(1-\pi(v | \theta)) n_\text{S}(v)\right] + \sum_{v \in \mathcal{F}} \log(1-\pi(v | \theta)) n_T(v)  \\
\end{align}

\begin{align}
  \ell(b | \theta)  &= \log(\bar{\pi}(b | \theta)) \sum_{v \in \mathcal{P}_b} n_\text{D}(v) + \log(1-\bar{\pi}(b | \theta)) \left(\sum_{v \in \mathcal{P}_b} n_\text{S}(v) +  \sum_{v \in \mathcal{F}_b} n_T(v)  \right).
\end{align}

Note that if we assume that the total number of combinations at each fixed site
is constant, e.g. $n_T = n_T(v)$ for all $v$, then we can use $\sum_v n_T(v) =
n_T |\mathcal{F}_b|$.

\section*{The B Components}

The core parts of our likelihood are,

\begin{align}
  \ell(\theta) &=  \sum_b [\log(\bar{\pi}(b | \theta)) Y_D(b) + \log(1-\bar{\pi}(b | \theta)) Y_S(b)]
\end{align}

where,

\begin{align}
  \bar{\pi}(b |\theta) = \pi_0(b) \bar{B}(b | \theta).
\end{align}

Here, $\bar{B}(b | \theta)$ is the predicted reduction in diversity due to BGS
in window $b$, given background selection parameters $\theta$. In practice,
this is site-specific. We can write the reduction at any neutral site $v$ in
the genome as the product of $B$s across all segments, 

\begin{align}
  B(v | \theta) = \exp\left(- \sum_g \int f(\mu(\mathcal{A}(g)), s, S_g) w(s|\mathcal{A}(g)) ds \right)
\end{align}

where $S_g$ is exogenous genomic data about the segment, $S_g = \{L_g, r_g,
\rho(|v-p_g|)\}$, where $L_g$ is the segment's length, $r_g$ is the
recombination rate per basepair in the segment, and $\rho(|v-p_g|)$ is the
recombination distance between the focal site $v$ and the segment position
$p_g$ (we approximate, and use the nearest end position to the neutral site).
Additionally, $w(s|\mathcal{A}(g))$ is the distribution of selection
coefficients for segment $g$, if segment $g$ is a member of annotation class
$\mathcal{A}(g)$.

We can think about the DFE as the conditional distribution of a particular
selection coefficient given a mutation occurs, for a particular annotation
class. The BGS function $f(\cdot)$ only depends on $\mu$ through the
introduction of deleterious alleles with selection coefficient $s$ at rate
$\omega(s|\mathcal{A}(g)) = \mu(\mathcal{A}(g)) w(s | \mathcal{A}(g))$. Thus,
we can write, 

\begin{align}
  B(v | \theta) = \exp\left(- \sum_g \int f(\omega(s|\mathcal{A}(g)), s, S_g) ds \right)
\end{align}


which we can discretize as,

\begin{align}
  B(v | \theta) = \exp\left(- \sum_g \sum_s f(\omega(s|\mathcal{A}(g)), s, S_g) \right).
\end{align}

Next, note that there are a finite number of annotation classes, $\mathcal{A}
\to \{a_1, a_2, \ldots, a_k\}$, so we can further partition this as

\begin{align}
  B(v | \theta) = \exp\left(- \sum_{\{g \;:\; \mathcal{A}(g) = a_1\}} \sum_s f(\omega(s| a_1), s, S_g) + \sum_{\{g \;:\; \mathcal{A}(g) = a_2\}} \sum_s f(\omega(s| a_2), s, S_g) + \ldots  \right)
\end{align}

Let us define the $d_\omega \times d_s \times d_g$ multidimensional array
$\mathbf{F}$, and the $d_g \times d_a$ feature classification matrix
$\mathbf{A}$.



% There are two approaches: interpolated $\pi(m | \theta)$ at the midpoint of the
% bin position $m$, and averaged bin-averaged $\bar{\pi}(m | \theta)$.

\section*{Windowed Diversity}

Although we use the components of diversity, $n_t$ and $n_s$, to calculate the
likelihood, it is still of interest to calculate diversity from these in a
window. The raw allele count data a $L \times \text{2}$ matrix, 

\section{B Scores}




\end{document}
