\documentclass[11pt]{article}
\RequirePackage{fullpage}
\RequirePackage[font=small,labelfont=bf]{caption}
\RequirePackage{amsmath,amssymb,amsthm}
\RequirePackage{graphicx}
\RequirePackage[hidelinks]{hyperref}
\RequirePackage{subcaption}
\RequirePackage{wasysym}
\RequirePackage{authblk}
\RequirePackage{bm}
\RequirePackage{bbm}
% \RequirePackage[osf]{mathpazo}
\RequirePackage[bibstyle=authoryear,citestyle=authoryear-comp,maxbibnames=9,maxcitenames=1,backend=biber,natbib=true,uniquelist=false,hyperref=true]{biblatex}
\usepackage{color}
\usepackage{nicefrac}

\renewcommand{\P}{\mathbb{P}}
\newcommand{\E}{\mathbb{E}}
\newcommand{\V}{\text{V}}
\DeclareMathOperator{\var}{Var}
\DeclareMathOperator{\cov}{cov}

\addbibresource{biblio.bib}

\title{Learning Theory Notes}

\begin{document}
\maketitle

\section*{Learning Limits}

\subsection*{The Simulation Sample $\bar{B}$ Estimator}

Our goal is to approximate the function $B = f(X)$ with some learned function
$\widehat{B} = \hat{f}(X)$ from evolutionary simulations. In each evolutionary
simulation, we sample some evolutionary parameters $X_k$ and evolve $r$
independent populations forward in time under these parameters and observe some
genealogy. From this genealogy, we estimate the reduction in neutral diversity
as:

\begin{align}
  \bar{B}_k = \frac{1}{r} \sum_{i=1}^r \frac{\hat{\pi}_{k,i}}{4N\mu}
\end{align}

there $\hat{\pi}$ is Tajima's estimator for pairwise diversity within a tree.
When we estimate $\hat{\pi}$ from using branch-mode tree statistics from an
observed genealogy, $\mu \to 1$, so we can ignore mutation.  If we take
expectation over the evolutionary process,

\begin{align}
  \E(\bar{B}_k) &= \frac{1}{4Nr} \sum_{i=1}^r \E(\hat{\pi}_{k,i}) \\
                &= \frac{T_k^{(2)}}{4N}  \\
                &= \frac{4B_kN}{4N}  \\
                &= B_k  \\
\end{align}

thus, the estimated reduction in diversity from simulations $\bar{B}$ is
unbiased, since $\E(\hat{\pi}_k) = 2T_k^{(2)} = 4B_k N$. For independent
evolutionary replicates, we calculate the variance of this estimator using
Tajima's equation for the variance of $\hat{\pi}$ as  

\begin{align}
  \var(\bar{B}_k) &= \frac{1}{16 N^2 r^2} \sum_{i=1}^r \var(\hat{\pi}_{k,i}) \\
                  &= \frac{\var(\hat{\pi}_{k,i})}{16 N^2 r} \\
                  &= \frac{1}{16N^2 r} \left( \frac{n + 1}{3(n-1)}\theta  + \frac{2(n^2 + n + 3)}{9n(n-1)}\theta^2 \right) \\
                  &= \frac{1}{16N^2 r} \left( \frac{n + 1}{3(n-1)} 4BN + \frac{2(n^2 + n + 3)}{9n(n-1)} 16B^2N^2 \right) \\
                  &= \underbrace{\frac{n + 1}{12(n-1)} \frac{B}{N r}}_\text{sampling noise} + \underbrace{\frac{2(n^2 + n + 3)}{9n(n-1)} \frac{B^2}{r}}_\text{evolutionary variance}
\end{align}

Now, let us look at the consistency of the estimator $\bar{B}$ (we drop the
parameter set $k$ for clarity) both in $r$ (over evolutionary replicates) and
in $n$ (as the sample size increases). Let $\bar{B}_r$ be the estimator of $B$
after $r$ evolutionary replicates (conditioning on some $n$). By Chebyshev's
inequality and for some $\epsilon > 0$,

\begin{align}
  \P\left(|\bar{B}_r - B| \ge \epsilon \right)  \le \frac{\var(\bar{B}_r)}{\epsilon^2}
\end{align}
%
and since $\lim_{r \to \infty} \var(\bar{B}_r) = 0$,

\begin{align}
  \lim_{r \to \infty} \P\left(|\bar{B}_r - B| \ge \epsilon \right)  = 0.
\end{align}

Thus, $\bar{B}_r \xrightarrow{p} B$ as $r \to \infty$ and $\bar{B}_r$ is
consistent in $r$. Now, let us look at the consistency of $\bar{B}_n$ in sample
size $n$. Note that $n \le 2N$ (i.e. our sample size is bounded by the number
of gametes in the population), so we imagine setting $n = 2N$ and taking the
limit $N \to \infty$,

\begin{align}
  \lim_{N \to \infty} \P\left(|\bar{B}_N - B| \ge \epsilon \right)  &\le \lim_{N \to \infty} \frac{\var(\bar{B}_N)}{\epsilon^2} \\
  \lim_{N \to \infty} \P\left(|\bar{B}_N - B| \ge \epsilon \right)  &\le \frac{2B^2}{9 r \epsilon^2}. \\
\end{align}

Thus, even if we sample the entire population, and let the population size $N
\to \infty$, $\bar{B}_n$ is still an inconsistent estimator in $n$. Intuitively
this is because 


In our case, we estimate pairwise diversity from the genealogical tree of the
entire population, so $n = 2N$

\begin{align}
  \var(\bar{B}_k) &= \frac{n + 1}{12(n-1)} \frac{B}{N r} + \frac{2(n^2 + n + 3)}{9n(n-1)} \frac{B^2}{r} \\
                  &= \frac{2N + 1}{12(2N-1)} \frac{B}{N r} + \frac{2(4N^2 + 2N + 3)}{18N(2N-1)} \frac{B^2}{r} \\
                  &\approx \frac{B}{12N r} + \frac{2N^2 + N}{9N^2} \frac{B^2}{r}
\end{align}

If we take the infinite population size limit,

\begin{align}
  \lim_{N \to \infty} \var(\bar{B}_k) &= \frac{2B^2}{9r}
\end{align}




\printbibliography

\end{document}
