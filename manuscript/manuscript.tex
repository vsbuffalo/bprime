\documentclass[11pt]{article}
\RequirePackage{fullpage}
\RequirePackage[font=small,labelfont=bf]{caption}
\RequirePackage{amsmath,amssymb,amsthm}
\RequirePackage{graphicx}
\RequirePackage[hidelinks]{hyperref}
\RequirePackage{subcaption}
\RequirePackage{wasysym}
\RequirePackage{authblk}
\RequirePackage{bm}
\RequirePackage{bbm}
% \RequirePackage[osf]{mathpazo}
\RequirePackage[bibstyle=authoryear,citestyle=authoryear-comp,maxbibnames=9,maxcitenames=1,backend=biber,natbib=true,uniquelist=false,hyperref=true]{biblatex}
\usepackage{color}
\usepackage{nicefrac}

\renewcommand{\P}{\mathbb{P}}
\newcommand{\E}{\mathbb{E}}
\newcommand{\V}{\text{V}}
% \DeclareMathOperator{\var}{var}
% \DeclareMathOperator{\cov}{cov}

\addbibresource{biblio.bib}

\title{}

\author{Vince Buffalo and Andrew Kern}

\begin{document}
\maketitle

A central goal of evolutionary genetics is to quantify the relative roles
different evolutionary processes have on genetic variation. Certain
evolutionary processes, such as demographic bottlenecks or expansions, act
uniformly across the genome as a whole. Other processes, such as selection, are
hypothesized to act differently across the genome depending on the density of
genomic features, and local heterogeneity in recombination and mutation rate.
With the emergence of population genomic data, a variety of statistical models
have teased apart the contribution of these various processes. These
statistical models differ in how they reduce large genomic data to summary
statistics, as well as in their statistical methodology. Some models reduce
genomic data to the histograms of the frequency spectrum (XXX), while others
model scalar summaries of the frequency spectrum along the genome. 

One class of models, parametric statistical models, leverage population genetic
theory to inform expectations of the mean pairwise diversity along the
chromosome. Estimation under these models often uses a composite likelihood
approach, where the joint distribution is assumed independent across positions.
While composite likelihoods are simpler to formulate and lead to consistent
parameter estimation, ignoring dependencies between neighboring sites has two
primary disadvantages. First, unlike proper likelihoods, the composite
likelihood's curvature is uninformative about parameter estimate uncertainty.
Previous work has circumvented this shortcoming through (1) resampling-based
uncertainty estimates, or (2) adjusting the composite likelihood's curvature to
account for these dependencies. Second, similar to inference under phylogenetic
comparative models, the dependency structure tells us how to properly weight
information across sites. For example, imagine a conserved segment 10 kilobases
long right on the border of two genomic bins, which due to linked selection,
bot experience a reduction in diversity. When estimating parameters, the
independence assumption of composite likelihoods would weight these observed
reductions equal to the amount of a reduction observed on an entirely different
independent chromosome. 

patterns of genetic variability along the genome are treated as
coming from an indeprocess $m(z)$.


Here, we focus on solving an issue processes: background

BGS as evolution around the static features of the genome. Due to the slow rate
of genomic structural evolution, much of the genome can be treated as static
throughout the genome. These regions are under differing levels of selective
constraint, which induces background selection at neighboring regions.




\printbibliography

\end{document}
