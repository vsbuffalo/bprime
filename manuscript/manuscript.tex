\documentclass[11pt]{article}
\RequirePackage{fullpage}
\RequirePackage[font=small,labelfont=bf]{caption}
\RequirePackage{amsmath,amssymb,amsthm}
\RequirePackage{graphicx}
\RequirePackage[hidelinks]{hyperref}
\RequirePackage{subcaption}
\RequirePackage{wasysym}
\RequirePackage{authblk}
\RequirePackage{bm}
% \RequirePackage{bbm}
\RequirePackage[osf]{mathpazo}
\RequirePackage[bibstyle=authoryear,citestyle=authoryear-comp,maxbibnames=9,maxcitenames=1,backend=biber,natbib=true,uniquelist=false,hyperref=true]{biblatex}
\usepackage{color}
\usepackage{nicefrac}

\renewcommand{\P}{\mathbb{P}}
\newcommand{\E}{\mathbb{E}}
\newcommand{\V}{\text{V}}
\DeclareMathOperator{\var}{var}
\DeclareMathOperator{\cov}{cov}

\addbibresource{biblio.bib}

\title{}

\author{Vince Buffalo and Andrew Kern}

\begin{document}
\maketitle

A central goal of evolutionary genetics is to quantify the relative roles
different evolutionary processes have on genetic variation. Certain
evolutionary processes, such as demographic bottlenecks or expansions, act
uniformly across the genome as a whole. Other processes, such as selection, are
hypothesized to act differently across the genome depending on the density of
genomic features, and local heterogeneity in recombination and mutation rate.
With the emergence of population genomic data, a variety of statistical models
have teased apart the contribution of these various processes. These
statistical models differ in how they reduce large genomic data to summary
statistics, as well as in their statistical methodology. Some models reduce
genomic data to the histograms of the frequency spectrum (XXX), while others
model scalar summaries of the frequency spectrum along the genome. 

One class of models, parametric statistical models, leverage population genetic
theory to inform expectations of the mean pairwise diversity along the
chromosome. Estimation under these models often uses a composite likelihood
approach, where the joint distribution is assumed independent across positions.
While composite likelihoods are simpler to formulate and lead to consistent
parameter estimation, ignoring dependencies between neighboring sites has two
primary disadvantages. First, unlike proper likelihoods, the composite
likelihood's curvature is uninformative about parameter estimate uncertainty.
Previous work has circumvented this shortcoming through (1) resampling-based
uncertainty estimates, or (2) adjusting the composite likelihood's curvature to
account for these dependencies. Second, similar to inference under phylogenetic
comparative models, the dependency structure tells us how to properly weight
information across sites. For example, imagine a conserved segment 10 kilobases
long right on the border of two genomic bins, which due to linked selection,
bot experience a reduction in diversity. When estimating parameters, the
independence assumption of composite likelihoods would weight these observed
reductions equal to the amount of a reduction observed on an entirely different
independent chromosome. 

patterns of genetic variability along the genome are treated as
coming from an indeprocess $m(z)$.


Here, we focus on solving an issue processes: background

BGS as evolution around the static features of the genome. Due to the slow rate
of genomic structural evolution, much of the genome can be treated as static
throughout the genome. These regions are under differing levels of selective
constraint, which induces background selection at neighboring regions.


Signal processing

Genomes have inherent structure that impacts the linked selection map. There is
tremendous heterogeneity across species in the organization of likely selected
regions on the genetic and physical maps.

Within an organism, there are two key determinants of B maps: recombination and
gene density. 

Initially, the relationship between recombination and diversity was regressed
with no regard to the spatial organization of the genome.

Certain regions are pinchpoints of selection: some combination of low
recombination and high gene density lead to very strong reductions in genetic
diversity. In these regions, coalescent dynamics change considerably; where as
under strong background selection, deleterious alleles would not fix, in these
regions the can fix. Additionally, the selective interference impacts positive
selection.

\section*{Theory}

\subsection*{Effective Population Size for Pairwise Diversity}

When quantifying genetic variation with average pairwise diversity, often a
single effective population size parameterizes the rate of pairwise
coalescences of the entire evolutionary history. This is sufficient for
constant-rate Kingman coalescences, but weak selection distorts genealogies
(desai and nicolaison), creating a coalescent process with non-constant rates.
Consequently, the expected pairwise coalescent time between two lineages $i$
and $j$ 

\begin{align}
  \E(T) &= \sum_i T(i) \P(\mathrm{MRCA}(i)) \\
        &= \sum_i T(i) \prod_{t=1}^i \left(1 - \frac{1}{2N_{e(i)}}\right)
\end{align}


% ADD TO INTRO TODO
Comment about characteristic scale $\nicefrac{\mu}{s}$. Is the fact that we see
BGS on megabase scales because this is the only scale we can detect strong
selection under?


approximately $\pi \approx 4N_e \mu$ (for low mutation rate). However, this
assumes a constant Kingman coalescent with coalescent effective population size
$N_e$. 

\subsection*{}

We model the reduction in pairwise coalescence times along the genome due to
background selection by extending the work of \textcite{Santiago2016-mu}
(hereafter, the SC16 model). The SC16 model derives from a class of linked
selection \parencite{Robertson1961-ho,Santiago1995-hx,Santiago1998-bs} that
predict the reduction in pairwise coalescence time at a focal neutral site
using macro-state parameters such as fitness variation, rather than micro-state
parameters like the frequency trajectories of selected alleles (e.g.
\textcite{Maynard_Smith1974-zr}). We will briefly introduce the SC16 model's
approach and how we validated this model could be extended to whole-genome
inference.

Classic background selection theory models how coalescence times are reduced by
deleterious mutation-selection balance at neighboring sites. Building upon
older models of genetic load \parencite{Crow1970-wj,Kimura1966-bk}, these
models consider the equilibrium distribution of the \emph{number} of
deleterious mutations segregating on haplotypes
\parencite{Charlesworth1993-gb,Nordborg1996-nq}. Given an infinitely large
population, or sufficiently strong selection, the deleterious mutations in
these models cannot fix; consequently, any lineage containing 

While the effect of selection on linked selection distorts neighboring
genealogies (XXX), skewing the site frequency spectrum (XXX) and altering
linkage disequilibria along the chromosome, we focus on the impact of selection
on pairwise coalescence rates. This simple summary is amenable to mathematical
and statistical inference, and is robust to difficulties in identifying rare
variants. The effect of background selection on pairwise coalescence rates is
expressed as the reduction ratio $B = \nicefrac{N_e}{N}$ where $N$ is the
drift-effective population size (i.e. including demographic processes) and
$N_e$ is the effective population size under background selection.

Under quantitative genetic models of linked selection, the reduction in
effective population size is determined by the product of fitness variation,
$V_f$, and an inflation factor $Q$. It is helpful to understand this by comparing
it to a model of $N_e$ without \emph{heritable} fitness variation but with
\emph{non-heritable} fitness variation. In this case, 

\begin{align}
  N_e = \frac{4N}{2 + V_k},
\end{align}

where $V_k$ is the variance in offspring number across individuals. Under a
constant population size Wright--Fisher model of reproduction, $V_k \approx 2$,
and this simplifies to $N_e = N$. However, as first noted by
\textcite{Robertson1961-ho}, linked selection inflates the variance in
offspring number, as a neutral allele that becomes associated with a
high-fitness or low-fitness genetic background has an $1-r$ chance of remaining
associated with that background in the next generation, where $r$ is the chance
of recombination (or independent assortment, $r=\nicefrac{1}{2}$). 


This persistence of this association inflates the effective variance in fitness
by a factory $Q^2$, 

\begin{align}
  N_e = \frac{4N}{2 + V_k + 4 Q^2 V_f},
\end{align}


\begin{align}
  N_e = \frac{4N}{2 + V_k + 4 Q(r)^2 V_f},
\end{align}




\begin{align}
  N_e = N \exp\left(-Q^2 V_f\right)
\end{align}

In a region of $L$ basepairs under purifying selection, there is a net flux of
deleterious mutations in at rate $U = 2 \mu L$ per generation (where the factor
of two is because in diploids, $2L$ is the total number of basepairs per genome). 

Typical models of background selection consider the equilibrium distribution of
the \emph{count} of deleterious mutations, rather than then level of fitness
variation implied by this count. If we assume fixed additive effects across
alleles and multiplicative fitness effects across loci, each copy of a
deleterious mutation an individual carries reduces their fitness by a factor of
$1-s$. The total reduction for an individual is

\begin{align}
  x = (1-s)^n
\end{align}

where $n$ is the number of deleterious alleles an individual carries. Under
mutation-selection balance, the population evolves to either a stationary
distribution under the strong BGS domain, or a traveling distribution in weak
selection case. Under some assumptions, equilibrium models of the strong domain
show that the distribution $f(n) \sim \text{Pois}(\nicefrac{U}{s})$
\parencite{Kimura1966-bk,Crow1970-wj} (CHECK) in the asexual case, and the
fixation rate of deleterious mutations per generation, or ratchet rate, is $R =
0$ since the probability of fixation, $p_F$ is zero. By contrast, as the
selection coefficient grows weaker, $s \to 0$, the probability of fixation $p_F
\to \nicefrac{1}{2N}$, and the ratchet approaches the neutral rate of $R =
\mu$. The weak selection domain is of importance because both because it
implies a certain likely irreversible rate of fitness reduction, and because it
could possibly confound population genetic inference which assumes dynamics
only under the strong model.

As $s \to 0$, the fitness distribution, put on a scale reflecting the fitness
relative at one moment in time, travels leftwards and changes shape. It has
been difficult to find dynamic equations for the rate of the ratchet and
fitness distribution, since lower moments of the distribution, such as the
variation, depend on higher moments, such as the skew
\parencite{Good2013-lp,Haigh1978-gt,Higgs1995-xc}. This is known as the ``moment
closure problem" and it occurs regardless of whether one models the count
number of deleterious mutations (XXX) or the fitness distribution (XXX).

We can approximate the mapping between the distribution of deleterious
mutations in the population, and the fitness distribution. It is worth noting
that the count of deleterious mutations in an individual $i$ can be written
$n_i = \sum_l^{L} g_l$ where each $g_l$ is an indicator variable for whether
the individual carries a mutation at position $l$. Then, $\V_i(n) = \sum_l
\V(g_l) + \sum_{i,j}\cov(g_i, g_j)$.

the
number of deleterious mutations in gamete $i$; thus $V(n) = \sum_i^{2N} V(x_i)
+ \sum_{i \ne j} \cov(x_i, x_j)$ where the last term is the correlation among
gamete mutation counts, which is an LD quantity. Then, assuming $s$ is small,
we can write express this in terms of the additive genetic variance, 

\begin{align}
  V(f) &\approx s^2 \V_a(n) + \delta_\text{LD}
\end{align}

The first term is the additive genic variance, which is 


Under this model, the rate of the ratchet $R = U  - s V(n)$, which implies,

\begin{align}
  R = U - \nicefrac{V(f)}{s} \\
  V(f) = s(U - R)
\end{align}



Throughout, it's worthwhile to keep into mind many population-level
macro-states are simply distributional summaries of the current population. For
example, the variance in deleterious mutations in the population is simply,
$\V(n) = \nicefrac{1}{2N} \sum_i (x_i - \mu_x)^2$





- We need to mention the two heterozygosities (S\&C '16 says they diverge under
$N_e s < 1$ -- could explain anything we see?)

- A point about reconciling the two micro and macro-state theories of
population and quantitative genetics.





\subsection*{The Rate of the Ratchet}


\section*{Methods}

\subsection*{Data Preprocessing}

\subsection*{Human}

Site-level neutral and accessibility masks

Bin-level masks based on fraction of accessible bases in a window (7\%).

Bin-level tail filtering.

\printbibliography

\end{document}
